FROM KUMAR 2009

Walkability is a general term that has different meaning depending on the context and field in which it is used.  Many different characteristics make a place ''walkable'', but all characteristics of walkable places and walkability in general all relate to built environment characteristics.  

In the health sciences, wlaking is found to have positive health benefits and built environments that promote walking, typically a result of more pedestrian friendly aesthetics that appeal to the senses and promote the utility of pedestrian travel are generally considered better for overall health (Sallis 2006, Shay et al 2003, Lovasi et al 2008).  

These characteristics can range from bustling town centers, to people oriented places, public spaces for people to congregate, mixed use environments, gridded streets, and proximit to schools, transit, parks, and retail all factor in to the walkability of a place.

This study had focused on the effects of the builit environment on the transportation choices of consumers.  In particular, the built evinromnet characteristics that have emerged under zoning regulations which are a result of political and planning processes.  While this study focuses on the resulting transportaiton behavior, many other benefits of optimal planning, particuarly zoning that promotes congestion relieving transportation can also be easiliy integrated with other concepts that promote walkabilit and give less prefernece to automobile transportation. 

Since automobile congestion is one of the societal problems that results from increasing use of transportation systems as cities continue to grow in population, changes to zoning that encourage non-auto transportation may have increased impact if such laws also incorporate best practices when dealing with the question of how to create attractive public spaces that promote pedestrian use and therefore are likely to increase interest in pedestrian modes of transportatoin as well.

These can include requirements such as builiding facing streets and garages or parking being hidden in alleys to create more attractive, less auto-oriented streetscapes that are more attractive to pedestrians

Mixed use can also play a big part in promoting both pedestrian lifestyles as well as vibrant businesses and streets.  A genearl rule of thumb is that common daily errands and most services are located within a ten minute walk time of residences.  Through alteration of the zoning code, promotion of smaller town centers within cities that ensures availabilit of goods and services within a short distnace of residences can be part of a city's zoning master plan and therefore help the city plan and promote the adaptation of the city towards more walkable city desing while meeting goals of congestion reduction by providing non-auto alternatives for everyday needs.

Ahwahnee Principles by New Urbanist arhcitects in 1991
proximitiy, connectivity, open spaces, building orientatoin (*Corbett 1994)

Southworth(2005) "Designing th Walkable City"
walkability “the built environment that supports and encourages walking by providing pedestrian comfort and safety, connecting people with varied destinations within a reasonable amount of time and effort, and offering visual interest in journeys throughout the network” - safety, comfort, presence of destinations, reduced distances, visual interest

Quality: Moudon et al. 2006
walkability extneds beyond ability to walk
sociability with neighbors - physical, mental, health benefits

Jacobs 1993: Great Streets
streets are magical, people want to be there

presence of people increases the notion of safety
more people, less chance of crime (jacobs 1969)

visually appealing - people are moving at a slower pace, therefore they want it to be visually appealing (lovasi 2008) - trees, landscaping, windows


Measure:  design and measure of sidewalks, meadians, safety standards, lighting, signals

transit oriented development - increase coverage and frequency of public transit to reduce auto dependence

Abbey 2005 - "extent to which walking is readily avialable as a safe, connected, accesible and pleasnt mode of transportation"

clifton 2006 - physical and natural environment that permits walking

schaeffer 1975 - people saw many advantages of living next to each other, especially access to people and amenities.  Before modern transport, slow transport necessitated close proximity

farming communities - southworth 2005

industrial era - increased transportation - reduction in transportation costs and living far from the city

Brignac 1999 - good road movement



ZONING

separate industrial and commerical uses from operating within residential areas to protect public health.

first comprehensive zoning ordinance - NYC 1916 (city and county of denver 2008)

federal highway act (Brignac 1999)

beginning of sprawl facilitated by cheap land and gas prices, good roads

new movements - new urbanism, traditional neighborhood development or neotraditional development, transit oreiented development, new community design, smart growth, sustainable communities, healthy communities (hemily 2004)

new urbanism (bohl 2000) - pedestrian oriented, human scaled neighborhood as the building block of communities
return to traditional town center, walkable streets, shopping centers, and grid street patterns, mixed land use

EPA 2004 - smart growth (knaap 2005)
www.newurbanism.org
Proximity of buildings to streets ? Presence of porches, doors and windows on the front side of buildings ? Tree lined streets ? On-street parking ? Hiding away of parking lots and garages to rear lane ? Narrow and slow speed streets
23
? Human scaled architecture ? Civic uses and sites within community ? Discernable center and edges ? Public spaces at centers ? Presence of Civic art
It

empirical : stonor 2003

shay 2003

natural environmental factors: topography, weather, time of day, day of week
built environment: land use pattern, distnace, connectivity, design of pathway, quality of pathway, safety
personal motivational: physical condition, family background, educaiton, attitudes and values, profession, cost (time and money)

Newman and Kenworthy 1996
emphasizes high density nad mixed use development aorund trnasit centers

cervero 1995 - mixed land use and residential density on commuting modal choice, distance,vehicle ownership
land use mix has greatest impact on walking
people prefer using cars after 300ft p.27

1200ft - weinstein 2008, moudon 2006

 zegras 2004 - land use mix in developing countries
 mixed use at origin increase walking, destination no impact
 
moudon 2006 - number of activities to induce walking
2 or more agglomerations of grocery stores ,food outlets and retial uses, more than 4 groceries reduces attractiveness to pedestrians.  office complexes more htan 9.8 acres or five schools within 3/4 mile deter walking. 

boarnet and crane 2000 - land use mix and density increase wlaking - focus on efficiency/cost reduction as important factor to increasing walking

cooper 2007 - importance of densificaiton - people's preferneces change as family demographics change

frank 1994 - population and employment density
pop density at origin - thresholds where mode is changed
increase pop density at origin increase walking
increase emp density at destination increase transit use

p.32



