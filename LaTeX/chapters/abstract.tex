The primary objective of the dissertation is to further existing research on the link between the built environment and travel behavior.  The dissertation proposes to make this advance in two distinct ways.  First, by testing the impact of land use regulation on travel behavior by incorporating zoning restrictions as an exogenous variable in the model.  Second, by explicitly modeling spatial variation in a multilevel discrete choice system of equations.  The dissertation is organized into three chapters.  The first is a literature review motivating the need for further refinement of existing models that address the relationship between travel behavior and the built environment.  The second develops a multilevel discrete choice system of equations that addresses unobserved travel preferences by incorporating multiple travel related decisions into a system of simultaneously determined choices related through stochastic error terms.  The third builds upon the second by explicitly modeling spatial dependence of choices in a similar system of equations and compares the results of both models to address the effect of spatial dependence on travel behavior-built environment model estimates.
