% Chapter 3: Implications
test

Location of economic activity has always been an important aspect of regional economics. Spatial proximity plays a key role in many decisions made by individuals when it comes to weighing the benefits and costs of purchase decisions, allocation of resources, and other economic behavior in general. Transportation choices are particularly affected by location, and distances between orgins and destinations undoubtedly influence the decision of travel behavior among individuals.
While the study of regional economics and regional science in general has been around for a long time, the development of formal econometric techniques to address location is a more recent development in the field. Of particular importance to the field of spatial econometrics is the treatment of spatial dependence (spatial autocorrelation) and spatial heterogeneity
(spatial structure). Spatial dependence and spatial heterogeneity are important in applied economic models because the presence of these phenomena may invalidate or bias mainstream results. In addition, these issues have been largely ignored in the mainstream literature. [2]
This study focuses on consumer choice behavior, in particular, transportation mode choice. Behavioral models of human choice follow the random utility model of choice behavior studied in the previous chapter. In this chapter, the previous travel behavior model is adapted to incorporate spatial dependence and spatial heterogeneity to test whether the results are significantly different from the standard econometric model of transportation mode choice where space is dealt with only informally through certain measurement variables of consumers.
38
It is particularly important in regards to land use restrictions, in that each individual faces a unique choice set of transportation choices based on their residential location and the proximity of this residential location to available goods, service, recreation, and employment opportunities.

